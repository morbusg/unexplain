%\XeTeXinputnormalization=1 % normalise to NFC form, using precomposed characters where possible instead base characters with combining marks.
\input fixinsert
\newdimen\frontmarginsep
\catcode`@=11
% folio (Fo): 12"*15", quarto (4to): 9.5"*12", octavo (8vo): 6"*9", duodecimo (12mo): 5"*7.0375
% PRIME!
% Unicode N??? Extended XeTeX Plain
% Vieth math aesthetics formulation scaling factors {7\over10} for 10pt,
% {5\over7} for 7pt {5\over5} for 5pt
\input mchardefs
\mcode`\,="6"1`,
\mcode`\.="6"1`.
\delimcode`\|="0"007C
\mcode`\|="0"0"007C
\delimcode`\<="0"027E8
\delimcode`\>="0"027E9
\mcode`\-="2"0"2212
\mcode`\/="0"0"2215
%\XeTeXdelcode`\/="0"2044
%\XeTeXmathcode`\/="0"0"2044
\mcode`\*="2"0"2217
\def\sqrt{\mradical"0"221A }
%\def\hat{\XeTeXmathaccent"7"0"002C6 } % Barbara's table uses the wide variants
%\def\check{\XeTeXmathaccent"7"0"002C7 }
%\def\bar{\XeTeXmathaccent"7"0"002C9 }
%\def\breve{\XeTeXmathaccent"7"0"002D8 }
\chardef\ttilde="007E
\def\tilde{\maccent"7"0"002DC }
%\def\widetilde{\maccent"7"0"0360 }
\def\ldots{\mathinner{\char"2026}}
\def\cdots{\mathinner{\char"22EF}}
\def\ddots{\mathinner{\char"22F1}}
\def\rddots{\mathinner{\char"22F0}}
\def\vdots{\mathinner{\char"22EE}}
%
\newfam\frakbffam\newfam\frakfam\newfam\scriptbffam\newfam\scriptfam\newfam\sansbfitfam\newfam\sansbffam\newfam\sansitfam\newfam\sansfam
%
\def\setMathType "#1" #2 {% font_name, font_size (in bp)
\ifdefined\loadedMathFont\else\gdef\loadedMathFont{#1}\fi%
\ifdefined\mathSize\else\gdef\mathSize{#2}\fi%
%
\def\hex##1{\ifcase##10\or1\or2\or3\or4\or5\or6\or7\or8\or9\or%
  A\or B\or C\or D\or E\or F\fi}%
%
\def\loadMathFont##1##2##3{% font name, options, family number
  \def\opts{script=math;mode=base}%
  \expandafter\font\csname##1\endcsname="#1:\opts;##2" at #2 true bp%
  \expandafter\font\csname##1s\endcsname="#1:\opts;+ssty0;##2" scaled 700%
  \expandafter\font\csname##1ss\endcsname="#1:\opts;+ssty1;##2" scaled 500%
  \textfont##3=\csname##1\endcsname%
  \scriptfont##3=\csname##1s\endcsname%
  \scriptscriptfont##3=\csname##1ss\endcsname}%
%
\loadMathFont{mathrm}{}{0}%
\ifdefined\testLatinModern\else\def\testLatinModern{Latin Modern Math}\fi%
\ifx\loadedMathFont\testLatinModern%
  \gdef\cong{\mathrel{\mathpalette\@vereq\sim}}% redef to plain
  \gdef\notin{\mathrel{\mathpalette\c@ncel\in}}% ditto
  \gdef\coleq{:=}%
  \mcode`\/="0"0"2044%
\fi%
% Italic
  \ifdefined\testEuler\else\def\testEuler{Neo Euler}\fi%
  % Neo Euler doesn't have italic, so dont map
  \ifx\loadedMathFont\testEuler%
    \textfont1=\mathrm\scriptfont1=\mathrms\scriptscriptfont1=\mathrmss%
    \mcode`\/="0"0"2044%
  \else%
    \loadMathFont{mathit}{mapping=italic}{1}%
  \fi%
% Calligraphic
  \ifdefined\testAsana\else\def\testAsana{Asana Math}\fi%
  \ifx\loadedMathFont\testAsana%
    % Asana uses Stylistic Alternates over script
    \loadMathFont{mathcal}{mapping=script;+salt}{2}%
  \else%
    % XITS and Neo Euler use Stylistic Set 1 over script
    \loadMathFont{mathcal}{mapping=script;+ss01}{2}%
  \fi%
\loadMathFont{mathbi}{mapping=bold italic}{3}%
\gdef\mib{\fam3}%
% Double-struck (aka Blackboard bold)
  \ifx\loadedMathFont\testEuler%
    \gdef\bb{\fam6}% Neo Euler doesn't have double-struck, so map it to bold
    % and some other stuff...
    \gdef\coloneq{:=}%
  \else%
    \loadMathFont{mathbb}{mapping=double-struck}{4}%
    \gdef\bb{\fam4}%
  \fi%
\loadMathFont{mathsl}{slant=0.2}{5}%
\loadMathFont{mathbf}{mapping=bold}{6}%
\loadMathFont{mathtt}{mapping=monospace}{7}%
% Fraktur
  \gdef\frakbf{\tenbf\fam\frakbffam}%
  \loadMathFont{mathfrakbf}{mapping=fraktur bold}{"\hex\frakbffam}%
  %
  \gdef\frak{\let\bf\frakbf\fam\frakfam}%
  \loadMathFont{mathfrak}{mapping=fraktur}{"\hex\frakfam}%
% Script
  \gdef\scrbf{\tenbf\fam\scriptbffam}%
  \loadMathFont{mathscrbf}{mapping=script bold}{"\hex\scriptbffam}%
  %
  \gdef\scr{\let\bf\scrbf\fam\scriptfam}%
  \loadMathFont{mathscr}{mapping=script}{"\hex\scriptfam}%
% Sans-serif
  \gdef\sfbfit{\fam\sansbfitfam}%
  \loadMathFont{mathsfbfit}{mapping=sans-serif bold italic}{"\hex\sansbfitfam}%
  %
  \gdef\sfbf{\tenbf\fam\sansbffam\let\it\sfbfit}%
  \loadMathFont{mathsfbf}{mapping=sans-serif bold}{"\hex\sansbffam}%
  %
  \gdef\sfit{\tenit\fam\sansitfam\let\bf\sfbfit}%
  \loadMathFont{mathsfit}{mapping=sans-serif italic}{"\hex\sansitfam}%
  %
  \gdef\sf{\let\bf\sfbf\let\it\sfit\fam\sansfam}%
  \loadMathFont{mathsf}{mapping=sans-serif}{"\hex\sansfam}%
  %
  % make it work both ways: \scr\bf or \bf\scr
  %\def\bf{\let\frak\frakbf\let\scr\scrbf\let\sf\sfbf\let\it\mib\tenbf\fam6 }
  %\def\it{\let\sf\sfit\let\bf\mib\tenit\fam1 }
  %
  \def\symbolfont{\textfont2}%
  \def\extensionfont{\textfont3}%
  %
  \def\mathsrulethickness{\fontdimen8}%
  \def\xheight{\fontdimen5}%
  \def\superscriptdrop{\fontdimen18}%
  \def\subscriptdrop{\fontdimen19}%
  \def\mathaxis{\fontdimen22}%
  %
  \thinmuskip=2.25mu% 3mu
  \medmuskip=3mu minus.75mu% 4mu plus2mu minus4mu
  \thickmuskip=4.5mu minus.9mu%minus0.9mu plus4.5mu % 5mu plus5mu
  %
  \delimitershortfall=\xheight\textfont0% 5pt
  \nulldelimiterspace=\fontdimen4\textfont0% 1.2pt
  \scriptspace=\fontdimen4\textfont0% 0.5pt
}
\def\{{\ifmmode\lbrace\else\char`\{\fi}%
\def\}{\ifmmode\rbrace\else\char`\}\fi}%
\let\|=\Vert
\let\to\rightarrow
\let\gets\leftarrow
\let\coleq\coloneq
\let\land\wedge
\let\lor\vee
\def\pmod#1{\allowbreak\;({\rm mod}\>#1)}
%\let\overline\overbar
\let\neq\ne
\def\iff{\;\Longleftrightarrow\;}
\def\setminus{\delim"2"0"005C } % Barbara's table uses smallsetminus
\def\underbrace{\maccent"7"0"23DF }
%\def\overline{\maccent"7"0"203E }
\def\overbrace{\maccent"7"0"23DE }
\def\underbrace{\maccent"7"0"23DF }
\def\macron{\maccent"7"0"035E }
%
%\mathcode‘\ ="8000 %\mathcode‘\(="4028 %\mathcode‘\+="202B %\mathcode‘\.="013A %\mathcode‘\;="603B %\mathcode‘\>="313E %\mathcode‘\\="026E %\mathcode‘\{="4266 %\mathcode‘\!="5021 %\mathcode‘\)="5029 %\mathcode‘\,="613B %\mathcode‘\/="013D %\mathcode‘\<="313C %\mathcode‘\?="503F %\mathcode‘\]="505D %\mathcode‘\|="026A %\mathcode‘\’="8000 %\mathcode‘\*="2203 %\mathcode‘\-="2200 %\mathcode‘\:="303A %\mathcode‘\=="303D %\mathcode‘\[="405B %\mathcode‘\_="8000 %\mathcode‘\}="5267
%
%\def\sc#1{{\tensc#1}}
\let\DS\displaystyle \let\TS\textstyle \let\SS\scriptstyle
\let\SSS\scriptscriptstyle
\let\hw\hidewidth
\def\eop{\hfill$\char"220E$}
\def\pover{\overwithdelims()}
\def\fover{\overwithdelims\lfloor\rfloor}
\long\def\proclaim #1. #2\par{%
  %\ifdim\lastskip=\smallskipamount\else\smallbreak\fi
  %\smallbreak%
  \begingroup%
  %\narrower%
  \noindent%
  \em{#1}.\enspace#2\par\endgroup%
  %\smallbreak\noindent%
  %\ifdim\lastskip<\smallskipamount
  %  \removelastskip\penalty55\smallskip\fi\noindent}
}
\newcount\licount
\long\def\ol#1\eol{\begingroup% keep stuff local for ol inside ol (inside ol...)
  \long\def\li{\advance\licount by1\item{\liform{\the\licount}}}%
  \ifdefined\liform\else\def\liform##1{##1.}\fi%
  \licount=0%
  \penalty10000 #1\smallskip\endgroup}
\def\super#1{%
  \char\ifcase#1"2070\or"00B9\or"00B2\or"00B3\or"2074\or"2075\or"2076
  \or"2077\or"2078\or"2079\fi}%
\def\sub#1{%
  \char\ifcase#1"2080\or"2081\or"2082\or"2083\or"2084\or"2085\or"2086
  \or"2087\or"2088\or"2089\fi}%
%\def\vfrac#1/#2{%\hbox{$\sup#1\fracslash\sub#2$}}
%  %\hbox{$\SS#1{\TS\fracslash}#2$}}
%  \raise.5ex\hbox{$\SS\m@th#1$}\hbox{%
%    $\m@th\mkern1mu\TS\fracslash\mkern1mu\SS#2$}}
\def\vfrac#1#2{%
  \def\num@r{#1}%
  \def\den@m{#2}%
  \mathpalette\vfr@k\relax}
\def\vfr@k#1#2{\raise.5ex\hbox{$\m@th #1\scriptscriptstyle\num@r$}%
  /\lower.5ex\hbox{$\m@th #1\scriptscriptstyle\den@m$}}%
\def\openupnormal{\afterassignment\@penupnormal\dimen@=}
\def\@penupnormal{\advance\normallineskip\dimen@
  \advance\normalbaselineskip\dimen@
  \advance\normallineskiplimit\dimen@}
\long\def\ul#1\eol{\begingroup%
  \ifdefined\liform\else\def\liform{•}\fi%
  \long\def\li##1{\item{\liform{}}##1}%
  #1\endgraf\endgroup\smallskip}
\def\edefappend#1#2{%
  \toks@ = \expandafter{#1}%
  \edef#1{\the\toks@ #2}%
}%
\def\centered{%
  \leftskip=0pt plus 1fil \rightskip=0pt plus 1fil
  \parskip=0pt \parfillskip=0pt plus0pt minus0pt\noindent}
\def\dmatrix#1{\null\,\vcenter{%\normalbaselines\m@th
  \ialign{\hfil$\displaystyle##$\hfil&&\quad\hfil$\displaystyle##$\hfil\crcr
    \strut\crcr\noalign{\kern-\baselineskip}
    #1\crcr\strut\crcr\noalign{\kern-\baselineskip}}}\,}
\def\caseswithdelim#1#2{\left#1\,\vcenter{\normalbaselines\m@th
  \ialign{$\mathstrut##\hfil$&\quad##\hfil\crcr#2\crcr}}\right.}
\def\bordermatrixwithdelims#1#2#3{\begingroup \m@th
  \setbox0=\vbox{\def\cr{\crcr\noalign{\kern2pt\global\let\cr=\endline}}
    \ialign{$##$\hfil\kern2pt\kern\p@renwd&\thinspace\hfil$##$\hfil
      &&\quad\hfil$##$\hfil\crcr
      \omit\strut\hfil\crcr\noalign{\kern-\baselineskip}
      #3\crcr\omit\strut\cr}}
  \setbox2=\vbox{\unvcopy0 \global\setbox1=\lastbox}
  \setbox2=\hbox{\unhbox1 \unskip \global\setbox1=\lastbox}
  \setbox2=\hbox{$\kern\wd1\kern-\p@renwd \left#1 \kern-\wd1
    \global\setbox1=\vbox{\box1\kern2pt}
    \vcenter{\kern-\ht1 \unvbox0 \kern-\baselineskip} \,\right#2$}
  \null\;\vbox{\kern\ht1\box2}\endgroup}
\def\cases#1{\left\{\,\vcenter{\m@th\ialign{%
  $##\hfil$&\quad##\hfil\crcr#1\crcr}}\right.}
%
\def\bcases#1{\caseswithdelim[{#1}}
\def\vcases#1{\caseswithdelim|{#1}}
\def\cbordermatrix#1{\bordermatrixwithdelims[]{#1}}
\def\bbordermatrix#1{\bordermatrixwithdelims\{\}{#1}}
\def\vbordermatrix#1{\bordermatrixwithdelims||{#1}}
\def\vmatrix#1{\left|\matrix{#1}\right|}
\def\bmatrix#1{\left[\matrix{#1}\right]}
\def\dpmatrix#1{\left(\dmatrix{#1}\right)}
\def\bra#1{\langle#1|}
\def\ket#1{|#1\rangle}
\def\braket#1#2{\langle#1|#2\rangle}
\def\displ@y{\global\dt@ptrue\m@th
  %\def\cr{\crcr\noalign{\vskip\baselineskip}}
  \everycr{\noalign{\ifdt@p\global\dt@pfalse%
    \else\penalty\interdisplaylinepenalty\fi}}}
\def\eqalign#1{\null\,\vcenter{\m@th%\displ@y
  \ialign{\strut$\hfil\displaystyle{##}$&$\displaystyle{{}##}\hfil$%
      \crcr#1\crcr}}\,}
\def\eqalignno#1{\displ@y \tabskip=\centering
  \halign to\displaywidth{\strut\hfil$\@lign\displaystyle{##}$\tabskip=0pt
    &$\@lign\displaystyle{{}##}$\hfil\tabskip=\centering
    &##\tabskip=0pt\crcr% removed \llap which was breaking \cr
    #1\crcr}}
\jot=0pt
\abovedisplayskip=\baselineskip \abovedisplayshortskip=0pt
\belowdisplayskip=\baselineskip \belowdisplayshortskip=\baselineskip
% TODO: footnotes don't always snap to grid
\newinsert\margins
\dimen\margins=\dimexpr\vsize-3\baselineskip
\skip\margins=0pt plus0pt minus0pt
\newtoks\verso
\newtoks\recto
\newdimen\frontmargin
\newdimen\spinemargin
\newdimen\pagewd
\newdimen\pageht
\pagewd=210mm
\pageht=297mm
\newbox\gridbox
\newif\ifspread
\newif\ifgrid
\newif\ifmatchxht
\chardef\dag"2020
\chardef\ddag"2021
\def\flushleft{\leftskip0pt plus1fil \rightskip0pt \parfillskip0pt\relax}
\def\flushright{\leftskip0pt \rightskip0pt plus1fil \parfillskip0pt plus1fil\relax}
\def\setType roman:"#1" mono:"#2" math:"#3" #4/#5*#6:#7 {
  \gdef\textSize{#4}%
  \setMathType "#3" #4
  \ifdefined\XeTeXprotrudechars%
    \XeTeXprotrudechars=2%
    \XeTeXdashbreakstate=1%
  \else%
    \input luaotfload.sty
  \fi%
  \def\setProtrusion##1{%
    %\lpcode##1 U`“ 800 \lpcode##1 U`” 900 \lpcode##1 U`" 800
    %
    \rpcode##1 U`” 150 \rpcode##1 U`’ 150 \rpcode##1 U`! 100%
    \rpcode##1 U`, 200 \rpcode##1 U`- 200 \rpcode##1 U`. 200%
    \rpcode##1 U`† 200 \rpcode##1 U`‡ 200%
    \rpcode##1 U`: 100 \rpcode##1 U`; 100 \rpcode##1 U`? 100}%
  \gdef\m/##1{\ifcase##1 0em \or1em \or.5em \or.3334em \or.25em \or.2em
    \or.1667em \or.1429em \or.125em \or.1111em \fi}
  \gdef\thinspace{\kern\m/6}
  % TODO: don't assume \font
  \def\interwordspace{\fontdimen2\font}%
  \def\interwordstretch{\fontdimen3\font}%
  \def\interwordshrink{\fontdimen4\font}%
  \def\extraspace{\fontdimen7\font}%
  \gdef\setSpace ##1+##2-##3++##4{%
    \interwordspace=##1\interwordstretch=\dimexpr##2-\interwordspace%
    \interwordshrink=\dimexpr\interwordspace-##3\extraspace=##4}%
  \def\setFont##1##2{%
    \expandafter\font\csname##1-roman\endcsname=
      "#1:mapping=tex-text;+onum" at ##2 true bp%
    \expandafter\font\csname##1-italic\endcsname=
      "#1/I:mapping=tex-text;+onum" at ##2 true bp%
    \expandafter\font\csname##1-slanted\endcsname=
      "#1:slant=0.2;mapping=tex-text;+onum" at ##2 true bp%
    \expandafter\font\csname##1-bold\endcsname=
      "#1/B:mapping=tex-text;+onum" at ##2 true bp%
    \expandafter\font\csname##1-bolditalic\endcsname=
      "#1/BI:mapping=tex-text;+onum" at ##2 true bp%
    \expandafter\font\csname##1-caps\endcsname=
      "#1:letterspace=#4;mapping=tex-text" at ##2 true bp%
    \expandafter\font\csname##1-smallcaps\endcsname=
      "#1:+smcp;+onum;mapping=tex-text" at ##2 true bp%
    \expandafter\font\csname##1-smallcaps-bold\endcsname=
      "#1/B:+smcp;+onum;mapping=tex-text" at ##2 true bp%
    \expandafter\font\csname##1-monospace\endcsname=
      "#2" at ##2 true bp%
    \expandafter\font\csname##1-monospace-italic\endcsname=
      "#2/I" at ##2 true bp%
    \expandafter\font\csname##1-monospace-bold\endcsname=
      "#2/B" at ##2 true bp%
    \expandafter\font\csname##1-monospace-bolditalic\endcsname=
      "#2/BI" at ##2 true bp%
    \gdef\raggedright{\rightskip=0pt plus2em
      \spaceskip=\m/4 \xspaceskip=\m/5\relax}%
    \ifdefined\XeTeXprotrudechars%
      \expandafter\setProtrusion\csname##1-roman\endcsname%
      \expandafter\setProtrusion\csname##1-italic\endcsname%
      \expandafter\setProtrusion\csname##1-slanted\endcsname%
      \expandafter\setProtrusion\csname##1-bold\endcsname%
      \expandafter\setProtrusion\csname##1-caps\endcsname%
      \expandafter\setProtrusion\csname##1-smallcaps\endcsname%
    \fi%
    {\csname##1-roman\endcsname\setSpace \m/4+\m/2-\m/5++\m/3}%
    {\csname##1-italic\endcsname\setSpace \m/4+\m/2-\m/5++\m/3}%
    {\csname##1-slanted\endcsname\setSpace \m/4+\m/2-\m/5++\m/3}%
    {\csname##1-bold\endcsname\setSpace \m/4+\m/2-\m/5++\m/3}%
    {\csname##1-caps\endcsname\setSpace \m/2+\m/2-\m/8++\m/3}%
    {\csname##1-smallcaps\endcsname\setSpace \m/4+\m/2-\m/5++\m/3}%
    %{\csname##1-monospace\endcsname\setSpace \m/1+0pt-0pt++0pt}%
    \expandafter\gdef\csname##1\endcsname{%
      \ifdefined\loadedMathFont\setMathType "\loadedMathFont"
        \number\numexpr##2-(\numexpr\textSize-\mathSize) \fi%
      \def\rm{\fam0 \csname##1-roman\endcsname}%
      \def\it{\fam1 \csname##1-italic\endcsname}%
      \def\sl{\fam5 \csname##1-slanted\endcsname}%
      \def\bf{\fam6 \csname##1-bold\endcsname}%
      \def\bfit{\fam3 \csname##1-bolditalic\endcsname}
      \def\sc{\csname##1-smallcaps\endcsname}%
      \def\tt{\fam7 \csname##1-monospace\endcsname\let\it\ttit\let\bf\ttbf}%
      \def\ttbf{\csname##1-monospace-bold\endcsname\let\it\bfttit}%
      \def\ttit{\csname##1-monospace-italic\endcsname}%
      \def\ttbfit{\csname##1-monospace-bolditalic\endcsname}%
      \def\caps{\csname##1-caps\endcsname}%
      \rm%
      %\setbox0\hbox{Ag}%
      %\setbox\strutbox\hbox{\vrule height\ht0 depth\dp0 width0pt}%
      \setbox\strutbox\hbox{\vrule height.7\baselineskip depth.3\baselineskip width0pt}%
      }}%
  \setFont{bodyfont}{#4}%
  \setFont{nonpareil}{6}%
  \setFont{minion}{7}%
  \setFont{brevier}{8}%
  \setFont{bourgeois}{9}%
  \setFont{garamond}{10}%
  \setFont{philosophy}{11}%
  \setFont{pica}{12}%
  \setFont{augustin}{14}%
  \setFont{greatprimer}{18}%
  \let\smalltext\brevier\let\longprimer\garamond\let\smallpica\philosophy%
  \bodyfont{}%
  %\emergencystretch=\m/5
  \widowpenalty=10000%
  \clubpenalty=0%\brokenpenalty=0 \finalhyphendemerits=0
  %\lastlinefit=200 % NOTE: nullifies \abovedisplayshortskip
  \baselineskip=#5 true bp \normalbaselineskip=\baselineskip%
  \lineskip=0pt \normallineskip=\baselineskip%
  \lineskiplimit=-\baselineskip \normallineskiplimit=-\baselineskip
  \abovedisplayskip=\baselineskip \abovedisplayshortskip=0pt%
  \belowdisplayskip=\baselineskip \belowdisplayshortskip=\baselineskip%
  \parindent=\dimexpr#6em/#5%
  \parskip=0pt%
  \topskip=\baselineskip%
  \smallskipamount=\baselineskip%
  \medskipamount=2\baselineskip%
  \bigskipamount=3\baselineskip%
  \insertskipamount=2\baselineskip%
  \special{papersize=\the\pagewd,\the\pageht}%
  \hsize=#6pc%
  \vsize=#7\hsize%
  \voffset=\dimexpr(\pageht-\vsize)/3-1in %+ \topskip
  \frontmargin=\dimexpr((\pagewd-\hsize)/3)*2-1in%
  \frontmarginsep=1em
  \spinemargin=\dimexpr(\pagewd-\hsize)/3-1in%
  \advance\hoffset by\dimexpr(\pagewd-\hsize)/2-1in%
  \ifdefined\gridcolor\else\def\gridcolor{1 .5 .5}\fi
  \ifgrid%
    \setbox\gridbox\line{%
      \special{color push rgb \gridcolor}%
      \vrule height\baselineskip width0pt
      \hrulefill
      \special{color pop}}
    \def\grid{\vbox to0pt{\vbox to\vsize{\leaders\copy\gridbox\vfil}}%
      \nointerlineskip}
  \fi%
  \def\pagebody{\ifgrid\grid\fi\vbox to\vsize{%
    \boxmaxdepth=\maxdepth\pagecontents\vss}}%
  \nopagenumbers%
  \frenchspacing%
  \def\makeheadline{\vbox to0pt{\vbox to-\dimexpr\baselineskip+\dp\strutbox{}\noindent\the\headline\vss}%
    %}
    \nointerlineskip}%
  \def\makefootline{\line{\the\footline}}%
  \skip\footins=\baselineskip%
  \verso={\hoffset=\frontmargin}%
  \recto={\hoffset=\spinemargin}%
  \def\makefrontmargin{%
    \ifspread%
      \ifodd\pageno\llap{\vbox to0pt{\box\margins\vss}\kern\frontmarginsep}% TODO: flushright/left ei (tietenkään) toimi, miten?
      \else\rlap{\kern\hsize\kern\frontmarginsep\vbox to0pt{\box\margins\vss}}\fi%
    \else\rlap{\kern\hsize\qquad\vbox to0pt{\box\margins\vss}}\fi}%
  \def\plainoutput{%
    \shipout\vbox{\makeheadline
      \makefrontmargin
      \pagebody
      \makefootline}
    \advancepageno
    \ifnum\outputpenalty>-20000 \else\dosupereject\fi}%
  \def\spreadoutput{\ifodd\pageno\the\verso\else\the\recto\fi\plainoutput}%
  \ifspread\output={\spreadoutput}\else\output={\plainoutput}\fi%
  }%
\def\abbr#1{{\sc\lowercase{#1}}}
\long\def\blockquote#1\eol{%
  \ifdim\lastskip=1sp\nobreak\else\smallskip\fi%
  \begingroup%
    \raggedright\narrower\noindent #1\par
  \endgroup\smallskip\noindent}
\def\quotee#1{\smallskip--- #1}
\def\bquote#{\bgroup\aftergroup\endgraf\raggedright\narrower\let\next= }
\newcount\sectonecount
\newcount\secttwocount
\newcount\sectthreecount
%\chapter\section\subsection\subsubsection
\def\emitsectglue#1{%
  \ifdim\lastskip=\baselineskip\removelastskip\fi
  \ifdim\lastskip=1sp\nobreak%
  \else%
    \vskip0pt plus#1\baselineskip
    \penalty-\numexpr#1*100+50\relax
    \vskip0pt plus-#1\baselineskip
    \vskip#1\baselineskip
  \fi}
\newtoks\postsectoks
\postsectoks={\nobreak\smallskip\noindent\kern-1sp\hskip1sp\relax}% the relax stops scanning, don't remove!
%
\newcount\sectioncount
\newcount\chaptercount
\newcount\subsectioncount
\newcount\subsubsectioncount
%
\newtoks\sectoks\sectoks={\noindent}
\newtoks\titlestyle\titlestyle={\eject\augustin\bf\the\sectoks}
\newtoks\subjectstyle\subjectstyle={\emitsectglue 2\pica\bfit\the\sectoks}
\newtoks\subsubjectstyle\subsubjectstyle={\emitsectglue 1\it\the\sectoks}
\newtoks\subsubsubjectstyle\subsubsubjectstyle={\emitsectglue 1\the\sectoks}
%
\long\def\title#1\par{%
  {\marks0{}\marks1{}\the\titlestyle#1}\marks0{#1}\the\postsectoks}
\long\def\subject#1\par{%
  {\marks1{}\the\subjectstyle#1}\marks1{#1}\the\postsectoks}
\long\def\subsubject#1\par{%
  {\the\subsubjectstyle#1}\the\postsectoks}
\long\def\subsubsubject#1\par{%
  {\the\subsubsubjectstyle#1}\the\postsectoks}
%
\def\thechapter{\the\count1.\quad}
\newtoks\titlestyle\titlestyle={\eject\augustin\bf\the\sectoks}
\long\def\chapter#1{%
  \global\advance\count1 by1
  \global\count2=0
  \global\count3=0
  \global\count4=0
  \title\thechapter #1}
%
\def\thesection{\ifnum\count1>0 \the\count1.\fi\the\count2.\quad}
\long\def\section#1{%
  \global\advance\count2 by1
  \global\count3=0
  \global\count4=0
  \subject\thesection #1}
%
\def\thesubsection{\ifnum\count1>0 \the\count1.\fi\the\count2.\the\count3\quad}
\long\def\subsection#1{%
  \global\advance\count3 by1
  \global\count4=0
  \subsubject\thesubsection #1}
%
\def\thesubsubsection{%
  \ifnum\count1>0 \the\count1.\fi\the\count2.\the\count3.%
  \the\count4.\quad}
\long\def\subsubsection#1{%
  \global\advance\count4 by1
  \subsubsubject\thesubsubsection #1}
%
\long\def\sect#1#2\par{%
  \ifdefined\sectform\else\def\sectform##1{##1}\fi%
  \ifdim\lastskip=1sp\else\emitsectglue{\numexpr4-#1}\fi\noindent%
  {\ifcase#1\relax\or\global\secttwocount=0\global\sectthreecount=0%
      \global\advance\sectonecount by1\ifspread\eject\fi%
      \sectform{\noindent\the\sectonecount.}%
    \or\global\sectthreecount=0\global\advance\secttwocount by1%
      \sectform{\noindent\the\sectonecount.\the\secttwocount.}%
    \or\global\advance\sectthreecount by1%
      \sectform{\noindent\the\sectonecount.\the\secttwocount.\the\sectthreecount.}%
    \else\garamond\it\fi%
    #2\nobreak\vskip\baselineskip\noindent\hskip-1sp\hskip1sp\relax}}
%\def\@ins{\par\begingroup\setbox0=\vbox\bgroup} % start a \vbox
%\def\endinsert{\egroup%
%  \if@mid\dimen@=\ht0\advance\dimen@ by\dp\z@
%    %\advance\dimen@ by\baselineskip% orig: 12\p@
%    \advance\dimen@ by\pagetotal\advance\dimen@ by-\pageshrink
%    \ifdim\dimen@>\pagegoal\@midfalse\p@gefalse\fi\fi
%  \if@mid\smallskip\box0\smallskip% orig: bigskip/-break
%  \else\insert\topins{\penalty100% floating insertion
%    \splittopskip=0pt\splitmaxdepth=\maxdimen\floatingpenalty=0
%    \ifp@ge\dimen@=\dp0\vbox to\vsize{\unvbox0 \kern-\dimen@}% depth is zero
%    \else\box0\medskip\fi}\fi\endgroup}% orig: bigskip
%\def\pagecontents{%
%  \ifvoid\topins \else\unvbox\topins\fi
%  \dimen0=\dp255 \unvbox255
%  \ifvoid\footins\else % footnote info is present
%    \vskip\skip\footins
%    %\footnoterule
%    \unvbox\footins\fi
%  \ifr@ggedbottom \kern-\dimen0 \vfil \fi}
\def\vfootnote#1{\insert\footins\bgroup
  %\baselineskip=.75\baselineskip
  \interlinepenalty=\interfootnotelinepenalty
  \splittopskip=\dimexpr\baselineskip+\ht\strutbox\relax %\ht\strutbox % top baseline for broken footnotes
  \splitmaxdepth=\dp\strutbox \floatingpenalty=20000
  %\leftskip=0pt \rightskip=0pt \spaceskip=0pt \xspaceskip=0pt
  \textindent{#1}\footstrut\futurelet\next\fo@t}
\def\footnoterule{\hrule width.3\hsize}
\def\footstrut{\vbox to\splittopskip{}}% orig splittopskip
\let\ts\thinspace
\def\em#1{{\it#1}}
\def\str#1{{\bf#1}}
{\obeyspaces\global\let =\ }
\def\obey{%
  \def\helper##1\yebo{%
    \smallskip##1\smallskip%\vskip\dp\strutbox}%
    \endgroup\noindent}%
  \begingroup\tt\obeylines\obeyspaces\helper}
%\def\displ@y{\lineskip=\baselineskip\lineskiplimit=0pt }
\def\displ@y{\global\dt@ptrue\m@th
  \def\cr{\crcr\noalign{\smallskip}}
    %\global\lineskip=\baselineskip\global\lineskiplimit=0pt }}}
  \everycr{\noalign{\ifdt@p\global\dt@pfalse%
    \else\penalty\interdisplaylinepenalty\fi}}}
\catcode`@=12
